\documentclass[12pt]{article}
\usepackage{titlesec}
\usepackage{listings}
\usepackage{color}

\definecolor{dkgreen}{rgb}{0,0.6,0}
\definecolor{gray}{rgb}{0.5,0.5,0.5}
\definecolor{mauve}{rgb}{0.58,0,0.82}

\lstset{frame=tb,
  language=Java,
  aboveskip=3mm,
  belowskip=3mm,
  showstringspaces=false,
  columns=flexible,
  basicstyle={\small\ttfamily},
  numbers=none,
  numberstyle=\tiny\color{gray},
  keywordstyle=\color{blue},
  commentstyle=\color{dkgreen},
  stringstyle=\color{mauve},
  breaklines=true,
  breakatwhitespace=true,
  tabsize=3
}

\setcounter{secnumdepth}{4}
\begin{document}
\title{%
  Homework 3 - PDF response file \\
  \large Internet of Things \\
  Politecnico di Milano \\
}

\author{Sara Acquaviva (943806), Davide Savoldelli (928676)}

\maketitle

\section{CoAP}
    \subsection{What’s the difference between the message with MID:
    3978 and the one with MID: 22636?}
    The packet with message id 3978 is "Confirmable", while the one with message id
    22636 is "Non Confirmable". This means that the former packet requires an acknowledgement for its reception,
    while the latter does not. In fact an ACK for the packet with mid 3978 is provided (the packet No. 6702).
    \subsection{Does the client receive the response of message No.
    6949?}
    Yes, the packet number 6949 is of the type "Confirmable", hence we can know if it has been received
    or not. As an ACK for the same MID exists (the packet No. 6953), the 6949 got an answer.
    \begin{lstlisting}
        frame.number == 6949
        coap.mid == 28357
    \end{lstlisting}
    Actually, the status of the ACK is "Method not Allowed". Hence, the GET does not 
    obatain a proper response, just an ACK.
    \subsection{How many replies of type Confirmable and result code
    “Content” are received by the server “localhost”?}
    There are eight packets that satisfy the filter.
    \begin{lstlisting}
        coap.code == 69 && ip.dst == 127.0.0.1
    \end{lstlisting}
\section{MQTT}
    \subsection{How many messages containing the topic
    “factory/department*/+” are published by a client with
    user name: “jane”? \small{(where * replaces the dep. number,
    e.g. factory/department1/+, factory/department2/+
    and so on.)}}
    First, we filter on the message sent by the username "jane".
    \begin{lstlisting}
      mqtt.username == "jane"
    \end{lstlisting}
    The are four messages of type "Connect", which connect to localhost from the TCP ports number 
    42821, 40989, 40005 and 50985.
    We use those ports to handle the connection to the username, as it appears only in Connect type messages.
    Filtering the "Publish" messages (type 0x11) which connect from the same ports and whose topic
    matches the regex "factory/department*/, we find six packets:
    \begin{lstlisting}
      mqtt.topic matches "factory/department[0-9]/" &&
      mqtt.msgtype == 3 && ip.src == 127.0.0.1 && 
      (tcp.srcport == 42821 || tcp.srcport == 40989 || tcp.srcport == 40005 || tcp.srcport == 50985) 
    \end{lstlisting}
    Being the packets of type "Published", obviously they cannot contain a topic with wildcards: these can be used only
    by subscribers.
    Unfortunately, none of the packet can be filtered by an eventual subscriber using the "+" wildcard after the department* level
    in the topic, because all of them contain an extra sublevel. To filter the packets a subscriber should specify a topic made like
    this: "factory/department*/+/+", being the "+" wildcard single level only.
    \subsection{How many clients connected to the broker “hivemq”
    have specified a will message?}
    First, we have to understand what IP address is related to the "hivemq" name. To do so, we 
    have a lookup on the DNS received packets.
    \begin{lstlisting}
      dns.qry.name contains "hivemq" && dns.flags.response && dns.a
    \end{lstlisting}
    After that, we can filter the packets sent to that IP address that match the former predicate.
    \begin{lstlisting}
      mqtt.conflag.willflag == 1 && (ip.dst == 3.120.68.56 || ip.dst == 18.185.199.22)
    \end{lstlisting}
    That said, there are 16 matching packets. Among them there are two packets with the same clientId but 
    different TCP source ports.
    \subsection{How many publishes with QoS 1 don’t receive the ACK?}
    We can filter all the publishes (129 packets) with:
    \begin{lstlisting}
      mqtt.qos == 1 && mqtt.msgtype == 3
    \end{lstlisting}
    Then, we filter only the Acknowledgements (74 packets) with:
    \begin{lstlisting}
      mqtt.msgtype == 4
    \end{lstlisting}
    Substracting the latter result from the former we obtain 55 packets without ACK.
    Among the Publishes, we want to select only the publisher-broker packets (77):
    \begin{lstlisting}
      mqtt.qos == 1 && mqtt.msgtype == 3 && tcp.dstport == 1883
    \end{lstlisting}
    Finally, we can conclude that 3 (77 - 74) of the 55 packets which unreceived ACK
    are from publisher-broker relationship, while the other 52 are from the broker-subscriber one. 
    \subsection{How many last will messages with QoS set to 0 are
    actually delivered?}
    First, filtering the packets which bring a Last Will Message, we notice that it starts with 
    the keyword "error".
    \begin{lstlisting}
      mqtt.conflag.willflag == 1
    \end{lstlisting}
    To know which Last Will Messages are actually delivered, we just look for the 
    delivered messages with QoS 1 starting with "error".
    \begin{lstlisting}
      mqtt.msg contains "error" && mqtt.qos == 1
    \end{lstlisting}
    There are 4 packets which satisfy the condition.
    \subsection{Are all the messages with QoS greater than 0 published by the
    client “4m3DWYzWr40pce6OaBQAfk” correctly delivered
    to the subscribers?}
    We must firstly find all the messages sent by “4m3DWYzWr40pce6OaBQAfk” with QoS greater than 0.
    Filtering the clientId:
    \begin{lstlisting}
      mqtt.clientid == "4m3DWYzWr40pce6OaBQAfk"
    \end{lstlisting}
    And looking at his socket, we filter his messages by QoS:
    \begin{lstlisting}
      mqtt.msgtype == 3 && ip.src == 10.0.2.15 && tcp.srcport == 58313 && mqtt.qos > 0
    \end{lstlisting}
    For everyone of them (there is only one message with QoS 2) we may check the reception of the PUBREC:
    \begin{lstlisting}
      mqtt.msgtype == 5&& ip.dst == 10.0.2.15 && tcp.dstport==58313
    \end{lstlisting}
    And then we check if the PUBREL has been sent back:
    \begin{lstlisting}
      mqtt.msgtype == 6&& ip.src == 10.0.2.15 && tcp.srcport == 58313
    \end{lstlisting}
    We find that there are no result out of the filter. Sticking to the schema QoS 2 (publisher-broker) + QoS 2 (broker-subscriber),
    we notice that the broker cannot proceed until the PUBREL has been received. Hence, we can infer that the message has not 
    been sent to the subscriber(s).
    In every case, the process is incomplete.

    \subsection{What is the average message length of a connect msg
    using mqttv5 protocol? Why messages have different
    size?}
    There are 63 packets of type "Connect" and MQTT version 5:
    \begin{lstlisting}
      mqtt.msgtype == 1 && mqtt.ver == 5
    \end{lstlisting}
    The average size of the packets is 91 bytes. They have different size because of the 
    optional fields that can be left empty. Moreover the MQTTv5 specifications
    explain that Metadata in the header of the message are available and they can be used to 
    specify user defined properties, which are, obviously, of different dimensions. 
    \subsection{Why aren’t there any REQ/RESP pings in the pcap?}
    The absence of the pings in the pcap is motivated by the fact that they are not required if 
    time specified in the Keep-Alive field (similar to HTTP's) of Connection packets is greater than
    the time required for the recipient(s) to answer. 

\end{document}