\documentclass[12pt]{article}
\usepackage{lingmacros}
\usepackage{tree-dvips}
\usepackage{titlesec}
\usepackage{amsmath}
\usepackage{hyperref}
\hypersetup{
    colorlinks,
    citecolor=black,
    filecolor=black,
    linkcolor=black,
    urlcolor=black
}



\setcounter{secnumdepth}{4}
\begin{document}
\title{%
  Formulario di Fisica Medica \\
  \large Facoltà di Medicina \\
  Università degli studi di Milano \\
}

\author{Davide Savoldelli}

\maketitle
\tableofcontents

\newpage
\section{Meccanica}
    \subsection{Cinematica}
        \subsubsection{Moto rettilineo uniforme}
            Velocità media: \[\vec{v_m} = \frac{\Delta s}{\Delta t}\]
            Velocità istantanea: \[\vec{v_i} = \lim_{\Delta t \to 0}{\frac{\Delta s}{\Delta t} = \frac{ds}{dt}}\]
            Legge oraria: \[\vec{x}(x) = x_0 + vt\]
        \subsubsection{Moto rettilineo uniformemente accelerato}
            Accelerazione media: \[\vec{a_m} = \frac{\Delta v}{\Delta t}\]
            Accelerazione istantanea: \[\vec{a_i} = \lim_{\Delta t \to 0}{\frac{\Delta v}{\Delta t} = \frac{dv}{dt}} = \frac{d^2s}{dt^2}\]
            Legge oraria: \[\vec{x}(t) = x_0 + v_0t + \frac{1}{2}at^2\]
            Velocità: \[\vec{v}(t) = v_0 + at\]
            \[v^2(x) = v_0^2 + 2ax \]
        \subsubsection{Moto armonico}
            Accelerazione legata alla posizione del punto:
            \[\vec{a}(t) = \frac{d^2}{dt^2}x(t) = -\omega^2x(t) \] 
            è un'equazione differenziale che si può risolvere con una funzione del tipo:
            Posizione: \[\vec{x}(t) = x_0 cos(\omega t + \phi)\]
            Velocità: \[\vec{v}(t) = -\omega x_0 sin(\omega t + \phi)\]
            Accelerazione: \[\vec{a}(t) = -\omega^2 x_0 cos(\omega t + \phi)\]
            Periodo: \[T = \frac{2\pi}{\omega}\]
            Frequenza: \[\nu = T^{-1} = \frac{\omega}{2\pi}\]
        \subsubsection{Moto circolare}
            Vettore raggio: \[\vec{r}(t) = Rx(t)\hat{i} + Ry(t)\hat{j} = Rcos(\theta(t))\hat{i} + Rsin(\theta(t)))\hat{j}\]
            Posizione: \[\vec{x}(t) = \theta(t)R\]
            Velocità (tangenziale): \[\vec{v}(t) = \omega(t) R\]
            Accelerazione tangenziale: \[\vec{a_t}(t) = \alpha(t) R\]
            Accelerazione centripeta: \[\vec{a_c}(t) = \frac{v^2(t)}{R} = \omega^2(t){R}\]
            Accelerazione: \[\vec{a}(t) = a_t(t)\hat{\tau} + a_c(t)\hat{n} \]         
        \subsubsection{Moto del proiettile}
            Equazioni del moto:
            
            \begin{equation*}
                \left\{
                \begin{array}{l}
                x(t) = x_0 + vt\\
                y(t) = y_0 + v_ot - \frac{1}{2}gt^2 
                \end{array}
                \right.
                \end{equation*}
    \subsection{Dinamica}
        \subsubsection{Leggi di Newton}
        \begin{itemize}
            \item Principio d'inerzia:  Un corpo non soggetto a forze permane nel suo stato di
            quiete o moto rettilineo uniforme. Condizione di equilibrio: \[\vec{R_{tot}} = 0\]
            \item Seconda legge di Newton: \[\vec{F} = m\vec{a}\]
            \item Principio azione-reazione: \[\vec{F}_{AB} = -\vec{F}_{BA}\]
        \end{itemize}
        \subsubsection{Forze}
            \paragraph*{Forza Peso}
            \[\vec{F}_p = - m\vec{g}\]
            \paragraph*{Forza Normale}
            Rappresenta la forza che un vincolo oppone a un corpo (secondo la terza legge della dinamica)
            Essa è perpendicolare alla superficie del vincolo.
            \paragraph*{Tensione}
            Rappresenta la forza che una corda tesa subisce e, se non ci sono deformazioni, \emph{trasmette costante per tutta la sua lunghezza}
            \paragraph*{Forza di attrito}
            \begin{itemize}
                \item Attrito statico e dinamico: \[\vec{F}_{att} = - \mu_{s/d}|\vec{N}|\]
                \item Attrito aerodinamico: \[|\vec{D}| = \frac{1}{2}C\rho A\vec{v^2}\]
            \end{itemize}
            \paragraph*{Forza centripeta}
            \[\vec{F}_c = m\frac{\vec{v^2}}{R}\]
            \paragraph*{Gravitazione}
            \begin{itemize}
                \item Forza Gravitazionale:
                \[\vec{F}_g = G\frac{m_1 m_2}{r^2}\]
                \item Velocità di fuga
                \[\vec{v} = \sqrt{\frac{2GM}{R}}\]
            \end{itemize}
            \paragraph*{Forza elastica (di Hooke)}
            \[\vec{F}_h = -k\Delta \vec{x}\]
        \subsubsection{Energia}
            \paragraph*{Lavoro}
            \[L = \int_{l}{\vec{F} \cdot d \vec{x}}\]
            \paragraph*{Lavoro con F costante}
            \[L = \int_{l}{\vec{F} \cdot d \vec{x}} = \vec{F}\int_{l}{ d \vec{x}} = \vec{F}(x_2 - x_1)\]
            \paragraph*{Esempio con F non costante (lavoro della forza elastica)}
            \[L = \int_{l}{-k x \cdot d \vec{x}} = -\frac{1}{2}k\Delta x^2\]
            \paragraph*{Energia potenziale}
            \[U = m\vec{g}h\]
            \paragraph*{Energia cinetica}
            \[K = \frac{1}{2}m\vec{v^2}\]
            \paragraph*{Teorema dell'energia cinetica}
            \[L_{TOT} = \Delta K\]
            \paragraph*{Conservazione dell'energia meccanica}
            \[\Delta E_m = \Delta U + \Delta K = 0 \text{\footnotesize{ (campi di forze conservative)}} \]
            \[\Delta E_m = \Delta U + \Delta K = L_{Fnc} \text{\footnotesize{ (campi di forze non conservative)}} \]
            \paragraph*{Potenza media}
            \[P_m = \frac{L}{\Delta t}\]
            \paragraph*{Potenza istantanea}
            \[P_i = \frac{dL}{dt} = \vec{F}d\vec{v}\]
        \subsubsection{Impulso e quantità di moto}
            \paragraph*{Quantità di moto}
            \[\vec{p} = m\vec{v}\]
            \paragraph*{Teorema dell'impulso}
            \[\vec{F} = m \vec{a} = \frac{md\vec{v}}{dt}\]
            \[\vec{F}dt = md\vec{v} = d\vec{p} = \vec{I}\]
            \paragraph*{Lavoro con F costante}
            \[L = \int_{l}{\vec{F} \cdot d \vec{x}} = \vec{F}\int_{l}{ d \vec{x}} = \vec{F}(x_2 - x_1)\]
            \paragraph*{Esempio con F non costante (lavoro della forza elastica)}
            \[L = \int_{l}{-k x \cdot d \vec{x}} = -\frac{1}{2}k\Delta x^2\]
            \paragraph*{Energia potenziale}
            \[U = m\vec{g}h\]
            \newpage
\section{Termodinamica}
    \subsection{Temperatura}
        \paragraph*{Dilatazione termica}
        \[V = V_0 (1 + \alpha T)\]        
    \subsection{Gas}
        \paragraph*{Equazione di stato dei gas perfetti}
        Termini macroscopici
        \[pV = nRT\]
        \paragraph*{Teoria cinetica dei gas}
        Termini microscopici
        \[pV = \frac{2}{3}n N_a \bar{K}\]
        da cui
        \[\bar{K} = \frac{3}{2}\frac{R}{N_a}T\]
        \paragraph*{Miscele di gas}
        Legge di Dalton
        \[P_{tot} = \sum_{i}^{n_{gas}}{P_i} \:\:\:\text{\footnotesize{con Pi pressione parziale del gas i-esimo nella miscela}}\]
        Legge di Henry
        \[c_i = \alpha \cdot p_i \:\:\:\text{\footnotesize{con alpha coefficiente di solubilità}}\]
    \newpage
        \subsection{Calore}
        \paragraph*{Flusso di calore}
        \[\Phi = \frac{Q}{A\Delta t}\]
        \paragraph*{Conduzione}
        \[\Phi = K_{cond}\frac{(T_a - T_b)}{L}\]
        \paragraph*{Convezione}
        \[\Phi = K_{conv} (T_b - T_a)\]
        \paragraph*{Irraggiamento}
        Potenza totale emessa (in Watt)
        \[H = e\sigma A T^4\]
        Flusso emesso
        \[\Phi = \frac{H}{A} = e\sigma T^4\]
        \subsubsection{Capacità termica e calore specifico}
        Capacità termica
        \[C = \frac{\Delta Q}{\Delta T}\]
        Calore specifico
        \[c = \frac{C}{m} = \frac{\Delta Q}{m \Delta T}\]
        Calore specifico molare
        \[c = \frac{C}{n} = \frac{\Delta Q}{n \Delta T}\]
        \paragraph*{Gas Perfetti}
        Legge di Mayer:
        \[c_v = c_p - R\]
        Gas monoatomici:
        \[c_v = \frac{3}{2} R\]
        \[c_p = \frac{5}{2} R\]
        Gas biatomici:
        \[c_v = \frac{5}{2} R\]
        \[c_p = \frac{7}{2} R\]
    \subsection{Primo Principio della Termodinamica}
        \[\Delta U_{sistema} = -\Delta U_{ambiente}\]
        \[\Delta U = Q_{entrante} - L_{uscente}\]
    \subsection{Trasformazioni}
        \subsubsection{Isocora}
        \[dU = n c_v dT\]
        \[L = 0 \:\:\: \text{\footnotesize{dV è nullo}}\]
        \[Q = dU = n c_v dT\]
        \subsubsection{Isobara}
        \[dU = n c_v dT\]
        \[L = P dV \]
        \[Q = n c_p dT\]
        \subsubsection{Isoterma}
        \[dU = 0 \:\:\: \text{\footnotesize{dT è nullo}}\]
        \[L = n R T_0 ln{\frac{V_2}{V_1}}\]
        \[Q = L = n R T_0 ln{\frac{V_2}{V_1}}\]
        \paragraph*{Funzione}
        Essendo la temperatura costante
        \[pV = nRT_0 = cost\]
        il grafico nel piano di Clapeyron è un'iperbole.
        \subsubsection{Adiabatica}
        \[dU = n c_v dT\]
        \[L = -dU = -n c_v dT\]
        \[Q = 0\]
        \paragraph*{Funzione}
        \[pV^{\gamma} = cost\]
        \[\gamma = \frac{c_p}{c_v}\]
        Dipendendo da gamma, il grafico è più inclinato dell'iperbole dell'isoterma.
    \subsection{Macchina di Carnot}
        Serie di Trasformazioni, in totale è una trasfomazione ciclica, quindi dU = 0, Q = L
        Poichè vi sono due isoterme, le temperature che variano non sono 4 (corrispondenti ai 4 stati) ma 2.
        \begin{itemize}
            \item Espansione isoterma A-B
            \[\text{Stato:}\:\:\:T_a = T_b = T_1, V_a < V_b, p_a > p_b\]
            \[dU = 0 \:\:\: \text{\footnotesize{dT è nullo}}\]
            \[L = n R T_1 ln{\frac{V_b}{V_a}}\]
            \[Q_1 = L = n R T_1 ln{\frac{V_b}{V_a}}\]
            \item Espansione adiabatica B-C
            \[\text{Stato:}\:\:\:T_b > T_c = T_2, V_b < V_c, p_b > p_c\]
            \[dU = n c_v (T_c - T_b)\]
            \[L = -dU = -n c_v d(T_c - T_b)\]
            \[Q = 0\]
            \item Compressione isoterma C-D
            \[\text{Stato:}\:\:\:T_c = T_d = T_2, V_c > V_d, p_c < p_d\]
            \[dU = 0 \:\:\: \text{\footnotesize{dT è nullo}}\]
            \[L = n R T_2 ln{\frac{V_d}{V_c}}\]
            \[Q_2 = L = n R T_2 ln{\frac{V_d}{V_c}}\]
            \item Compressione adiabatica D-A
            \[\text{Stato:}\:\:\:T_d < T_a = T_1, V_d > V_a, p_d < p_a\]
            \[dU = n c_v (T_a - T_d)\]
            \[L = -dU = -n c_v (T_a - T_d)\]
            \[Q = 0\]
        \end{itemize}
        \subsubsection{Rendimento della macchina di Carnot}
        Rendimento:
        \[\eta = \frac{L_{tot}}{Q_{assorbito}} \]
        \[L_{tot} = nR(T_1-T_2)ln{\frac{V_b}{V_a}} = Q_{tot}\]
        \[L_{tot} = Q_{tot} =  Q_1 + Q_2 < Q_1 \:\:\:\text{\footnotesize{essendo Q2 negativo}}\]
        \[\eta = \frac{Q_1 + Q_2}{Q_1} = 1 + \frac{Q_2}{Q_1} \:\:\:\text{\footnotesize{Valida per tutte le trasfomazioni cicliche}} \]
        Con il ciclo di Carnot, in particolare:
        \[\eta = 1 + \frac{T_2}{T_1} \:\:\:\text{\footnotesize{Sostituisco a Q1 e Q2 quelli trovati per la macchina di Carnot}} \]
    \subsection{Entropia e Secondo Principio}
        L'entropia è una funzione di stato, perciò
        \[\Delta S = S_f - S_i\]
        \[\Delta S = \int_{i}^{f} \frac{\delta Q}{T}  \:\:\:\text{\footnotesize{trasformazione 
        reversibile}}\]
        \[\Delta S > \int_{i}^{f} \frac{\delta Q}{T}  \:\:\:\text{\footnotesize{trasformazione 
        irreversibile}}\]
        Esempio:
        Considero l'espansione libera di un gas in un sistema isolato, esso è un processo irreversibile, 
        per valutarlo devo connettere gli stati inziale e finale con una trasformazione reversibile, per esempio un'isoterma reversibile.
        \newline
        Calcolo l'entropia:
        \[\Delta S_{sistema} = \int_{i}^{f} \frac{\delta Q}{T} = \frac{1}{T} \int_{i}^{f}{\delta Q} = \frac{1}{T} \int_{i}^{f}{\delta L} = \frac{1}{T} \int_{i}^{f}{pdV} \]
        \[= \frac{1}{T} \int_{i}^{f}{\frac{nRTdV}{V}} = \frac{1}{T} nRT ln{\frac{V_f}{V_i}} = nRln{\frac{V_f}{V_i}} > 0\]
        Calcolo ciò che succede per l'universo (sistema + ambiente):
        \[\Delta S_{sistema} > 0\]
        \[\Delta S_{universo} = \Delta S_{sistema} + \Delta S_{ambiente} \]
        Ma il sistema è isolato, quindi non ha variazione di entropia, perciò:
        \[\Delta S_{universo} = \Delta S_{sistema} > 0\]
\newpage
    \section{Fluidi}
    \subsection{Fluidostatica}
    \paragraph*{Legge di Stevino}
    \[P(h) = P_0 + \rho_{F} g h\]
    \[\Delta P = \rho_{F} g h\]
    \paragraph*{Principio di Pascal}
    \[\frac{F_1}{A_1} = \frac{F_2}{A_2} = \Delta P\]
    \paragraph*{Principio di Archimede}
    \[F_{arch} = \rho_{F} g V\]
    \subsection{Fluidodinamica}
    \paragraph*{Portata}
    \[Q = \frac{\Delta V}{\Delta t}\]
    \paragraph*{Equazione di continuità}
    \[Q = Av\]
        \subsubsection{Fluidi Ideali}
        \paragraph*{Teorema di Bernoulli}
        \[\rho g h_1 + P_1 + \frac{1}{2} \rho v_1^2 = \rho g h_2 + P_2 + \frac{1}{2} \rho v_2^2 = k\]
        \paragraph*{Paradosso di Bernoulli}
        \emph{La pressione è maggiore dove la sezione è maggiore.} 
        Oseervando l'equazione di Bernoulli, P è maggiore 
        quando v è minore (a parità di tutti gli altri contributi ed essendo la somma di questi costante).
        Perciò se v è minore per l'equazione di continuità l'area
        è maggiore, considerando un fluido stazionario (a portata costante).
        \paragraph*{Teorema di Torricelli}
        \[v_{uscita} = \sqrt{2gh}\]
        \subsubsection{Fluidi Reali}
        \[\rho g h_1 + P_1 + \frac{1}{2} \rho v_1^2 = \rho g h_2 + P_2 + \frac{1}{2} \rho v_2^2 + R\]
        \[\]
        \paragraph*{Gradiente di velocità}
        \[lim_{\Delta r \to 0}{\frac{(v + \Delta v)}{\Delta r}} = \frac{dv}{dr}\]
        \paragraph*{Sforzo tangenziale}
        \[\frac{F_{att}}{A} = \eta \frac{dv}{dr}\]
        \[F_{att} = \eta A \frac{dv}{dr}\]
        \paragraph*{Legge di Poiseuille (regime laminare)}
        \[v(r) = \frac{1}{4}\frac{P_1 - P_2}{l \cdot \eta}(R^2 - r^2)\]
        \[v_{max} = \frac{1}{4}\frac{\Delta P}{l \cdot \eta}R^2\]
        \[v(r) = v_{max}(1-\frac{r^2}{R^2})\]
        \paragraph*{Calcolo portata, data la velocità in un condotto circolare}
        \[Q = \frac{\pi}{8}\frac{\Delta P}{l \cdot \eta}R^4\]
        \[\Delta P = \mathcal{R}Q\]
        \[\mathcal{R} = \frac{8}{\pi}\frac{l \cdot \eta}{R^4}\]
        Analogia con la resistenza elettrica:
        \begin{itemize}
            \item Per condotti in parallelo: \[\frac{1}{\mathcal{R}_{eq}} = \frac{1}{\mathcal{R}_{1}} + \frac{1}{\mathcal{R}_{2}} + ... + \frac{1}{\mathcal{R}_{n}}\]
            \[\mathcal{R}_{eq} = \frac{1}{\sum_{i = 1}^{n}{\frac{1}{\mathcal{R}_i}}}\]
            \item Per condotti in serie: \[\mathcal{R}_{eq} = \mathcal{R}_1 + \mathcal{R}_2 + ... + \mathcal{R}_n = \sum_{i = 1}^{n}{\mathcal{R}_i}\]
        \end{itemize}
        \paragraph*{Regime turbolento (Re = 2000)}
        Velocità critica di passaggio da regime laminare a turbolento
        \[v_{crit} = \frac{Re \cdot \eta}{R \cdot \rho}\]
        \paragraph*{Fluidi Newtoniani}
        La viscosità dipende solo dalla temperatura
        \paragraph*{Fluidi non Newtoniani}
        La viscosità dipende dalla temperatura e dal gradiente di velocità del fluido.
        Si distinguono in fluidi:
        \begin{itemize}
            \item pseudoplastici, per i quali la viscosità diminuisce all'aumentare del grandiente di velocità (come il sangue)
            \item dilatanti, per i quali la viscosità aumenta all'aumentare del gradiente di velocità
        \end{itemize}
    \subsection{Tensione superficiale}
    L'aumento di superficie di un fluido è possibile agendo \emph{contro} le forze di coesione che 
    uniscono le molecole del liquido. Un fluido infatti tende sempre ad avere una forma che gli consenta di
    esporsi il meno possibile ad un fluido di diversa natura.
    \paragraph*{Forza necessaria ad espandere la superficie}
    \[F = 2\tau l \:\:\:\:\tau \: \text{detta tensione superficiale e dipendente dal fluido}\]
    \paragraph*{Lavoro per unità di superficie}
    \[L_c = 2\tau l \Delta x = 2 \tau \Delta S\]
    In una superficie sferica:
    \[\Delta L_c = F\delta s = P_c \delta A \Delta R\]
    \[L_{tot} = P_c \Delta R S_{tot} = P_c \Delta R 4\pi R^2\]
    possiamo esprimere il lavoro anche come:
    \[\Delta S = 4\pi (R + \Delta R)^2 - 4\pi R^2\]
    \[\Delta S = 4\pi R^2 + 4\pi \Delta R^2 + 8\pi R\Delta R - 4\pi R^2\]
    \[4\pi \Delta R^2 \:\:\:\: \text{trascurabile}\]
    Perciò calcolo il lavorro usando il nuovo differenziale di superficie:
    \[L = 2 \tau \Delta S = 16 \tau\pi R \Delta R\]
    uguagliando con il lavoro trovato in precendenza
    \[16 \tau\pi R \Delta R =  P_c \Delta R 4\pi R^2\]
    trovo la Pressione di Laplace.
    \begin{itemize}
    \item per una superfice sferica:
    \[P_c = \frac{4\tau}{R}\]
    \item con una sola interfaccia:
    \[L_c = 2 \tau \Delta S\]
    \[P_c = \frac{2\tau}{R}\]
    \item per una superfice ellissoidale:
    \[P_c = \tau(\frac{1}{R_1}\cdot\frac{1}{R_2})\]
    \item per una superfice cilidrica:
    \[P_c = \tau\frac{1}{R}\]
    \end{itemize}
    \section{Elettromagnetismo}
    \subsection{Elettrostatica}
    \paragraph*{Campo elettrico}
    \[\vec{E} = k \frac{Q}{r^2} \cdot \hat{r}\]
    \[k = \frac{1}{4\pi\epsilon}, \:\:\:\: \epsilon = \epsilon_0\epsilon_r, \:\:\:\: \hat{r} \:\text{\footnotesize{versore raggio di modulo 1}}\]
    \paragraph*{Forza elettrostatica}
    \[\vec{F}_{el} = k \frac{Q q}{r^2} \cdot \hat{r}\]
    \[\vec{F}_{el} = q\vec{E}\]
    \paragraph*{Lavoro forza elettrostatica}
    \[L = -k \frac{Q q}{r} \]
    \paragraph*{Energia potenziale elettrostatica}
    \[W(r) =-L= k \frac{Q q}{r} \]
    \paragraph*{Potenziale elettrostatico}
    \[\Delta V = V_b - V_a = -\frac{L_{a,b}}{q} = k\frac{Q}{r}\]
    \[V(r) = \frac{W(r)}{q} = - \frac{L_{\infty, r}}{q}\]
    \paragraph*{Momento dipolo elettrico}
    \[P = qd\]
    \newline
    \subsection{Campo elettromagnetico}
    \paragraph*{Corrente elettrica}
    \[I(t) = \frac{dq}{dt}\]
    \paragraph*{Prima legge di Ohm}
    \[V = \mathcal{R}I\]
    \paragraph*{Seconda legge di Ohm}
    \[\mathcal{R} = \rho\frac{L}{S}\]
    \[\rho(T) = \rho_0(1+\alpha T)\]
    \paragraph*{Forza elettromotrice}
    Lavoro che il campo elettromotore compie per far percorrere
    ad una carica unitaria positiva l'intero giro del circuito.
    Rappresenta la ddp quando il generatore non eroga corrente (ma solo tensione).
    \paragraph*{Leggi di Kirchhoff}
    \begin{itemize}
        \item Legge dei nodi \[\sum_{j} I_{j, entranti} = \sum_{k} I_{k, uscenti}\]
        \item Legge delle maglie \[\sum_{i} \Delta V_{i, maglia} = 0\]
    \end{itemize}
    \paragraph*{Resistenze}
    \begin{itemize}
        \item Resistenze in parallelo: \[\frac{1}{\mathcal{R}_{eq}} = \frac{1}{\mathcal{R}_{1}} + \frac{1}{\mathcal{R}_{2}} + ... + \frac{1}{\mathcal{R}_{n}}\]
        \[\mathcal{R}_{eq} = \frac{1}{\sum_{i = 1}^{n}{\frac{1}{\mathcal{R}_i}}}\]
        \item Resistenze in serie: \[\mathcal{R}_{eq} = \mathcal{R}_1 + \mathcal{R}_2 + ... + \mathcal{R}_n = \sum_{i = 1}^{n}{\mathcal{R}_i}\]
    \end{itemize}
    \paragraph*{Potenza elettrica}
    \[L = \Delta Q \]
    \[P = \frac{L}{\Delta t} = \frac{VQ}{\Delta t} = VI\]
    \[P = \frac{V^2}{\mathcal{R}} = \mathcal{R}I^2\]
    \paragraph*{Capacità}
    \[C = \frac{Q}{\Delta V}\]
    Capacità conduttore sferico
    \[C = 4\pi \epsilon_0r\]
    \subsubsection{Condensatori}
    \paragraph*{Campo elettrico}
    \[\sigma = \frac{Q}{A}\]
    \[\vec{E} = \frac{\sigma}{\epsilon}\]
    \paragraph*{Potenziale condensatore piano}
    \[\Delta V = \frac{L}{Q} = \vec{E}d\]
    \paragraph*{Capacità condensatore piano}
    \[C = \epsilon \frac{A}{d}\]
    \paragraph*{Lavoro necessario a spostare una carica}
    \[L_{tot} = \frac{1}{2}\frac{Q^2}{C} = \frac{1}{2}V^2C = \frac{1}{2}QV\]
    \begin{itemize}
        \item Condensatori in serie: \[\frac{1}{C_{eq}} = \frac{1}{C_{1}} + \frac{1}{C_{2}} + ... + \frac{1}{C_{n}}\]
        \[\mathcal{R}_{eq} = \frac{1}{\sum_{i = 1}^{n}{\frac{1}{C_i}}}\]
        \item Condensatori in parallelo: \[C_{eq} = C_1 + C_2 + ... + C_n = \sum_{i = 1}^{n}{C_i}\]
    \end{itemize}
    \paragraph*{Legge di carica del condensatore (in circuiti RC)}
    \[I_c(t) = I_0\cdot e^{-\frac{t}{RC}}\]
    \[V_c(t) = \frac{dI_c(t)}{dt} = V_0\cdot(1-e^{-\frac{t}{RC}})\]
    \section{Onde}

Don't forget to include examples of topicalization.
They look like this:

{\small
\enumsentence{Topicalization from sentential subject:\\ 
\shortex{7}{a John$_i$ [a & kltukl & [el & 
  {\bf l-}oltoir & er & ngii$_i$ & a Mary]]}
{ & {\bf R-}clear & {\sc comp} & 
  {\bf IR}.{\sc 3s}-love   & P & him & }
{John, (it's) clear that Mary loves (him).}}
}

\subsection*{How to handle topicalization}

I'll just assume a tree structure like (\ex{1}).

{\small
\enumsentence{Structure of A$'$ Projections:\\ [2ex]
\begin{tabular}[t]{cccc}
    & \node{i}{CP}\\ [2ex]
    \node{ii}{Spec} &   &\node{iii}{C$'$}\\ [2ex]
        &\node{iv}{C} & & \node{v}{SAgrP}
\end{tabular}
\nodeconnect{i}{ii}
\nodeconnect{i}{iii}
\nodeconnect{iii}{iv}
\nodeconnect{iii}{v}
}
}

\subsection{Mood}

Mood changes when there is a topic, as well as when
there is WH-movement.  \emph{Irrealis} is the mood when
there is a non-subject topic or WH-phrase in Comp.
\emph{Realis} is the mood when there is a subject topic
or WH-phrase.

\end{document}