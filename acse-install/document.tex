\documentclass{article}
\usepackage{graphicx}
\usepackage[os=win]{menukeys}
\usepackage{hyperref}
\usepackage{fontawesome}
\makeatletter
\tw@make@key@box{OS@mac}{\faApple}
\tw@make@key@box{OS@win}{\faWindows}
\tw@make@key@macro*{\OS}
\tw@make@key@box{capslock@win}{\textsf{CapsLock}}
\tw@make@key@box{capslock@mac}{\textsf{caps lock}}
\makeatother

\begin{document}

\title{ACSE Quick installation guide}
\author{Davide Savoldelli}

\maketitle

\section{Requirements}
\begin{itemize}
    \item Windows 10 build 14393 or above.
    \item The ACSE source code version 1.1.5 or above. This can be provided by your lecturer on the appropriate channels (e.g: BeeP).
    \item \href{https://code.visualstudio.com/}{Microsoft Visual Studio Code}
\end{itemize}

\section{Installation of Windows Subsystem for Linux (WSL) and first setup}
In order to provide a clean installation of the required binaries for the compiler to work, without the need to install 
a complete Virtual Machine on the System or to use the dual boot, the best option is to opt for
the WSL. Windows Subsystem for Linux is a lightweight version of a Linux distro under Windows 10 environment.
With it, the POSIX syscalls of the virtualized machine are on-the-fly translated into Windows NT system calls.
\begin{enumerate}
    \item Press \keys{\OSwin + R}, type \verb'powershell' and press \keys{\ctrl + \shift + \enter}. Insert the Administrator password if prompted.
    \item Type: 
    \begin{scriptsize}{\begin{verbatim}Enable-WindowsOptionalFeature -Online -FeatureName Microsoft-Windows-Subsystem-Linux\end{verbatim}}\end{scriptsize}
    and press \keys{\enter}. Reboot the computer if asked.
    \item Go to the Microsoft Store Application and install a packaged Linux distro. \href{https://www.microsoft.com/it-it/p/pengwin/9nv1gv1pxz6p}{Pengwin} or \href{https://www.microsoft.com/it-it/p/ubuntu/9nblggh4msv6}{Ubuntu} are recommended.
    Once opened the previous installed Linux distro and waited the environment to set up, choose a UNIX username and a password.
    \item Install the required packages: 
    \begin{verbatim}sudo apt update && sudo apt install gcc flex bison\end{verbatim}
    \item In Windows, open Visual Studio Code and install the \href{https://marketplace.visualstudio.com/items?itemName=ms-vscode-remote.remote-wsl}{Remote - WSL extension}. This will prevent you from installing \verb+code+ under Linux environment.
    \item Download the ACSE archive and unzip it into a handy Linux directory (for example \verb+\home\<YOUR_USER_NAME>+). \newline \newline \begin{small} Note: in Windows you can use the builtin 9P Virtual Server to access Linux directories. Open Explorer and type \verb+\\wsl$\<YOUR_DISTRO_NAME>+ in the search bar (for example: \verb+\\wsl$\Ubuntu+) and press \keys{\enter}.\end{small}
    \item Open the Linux distro App, enter the command \verb+nano $HOME/.bashrc+ then add the follwing line to the file:
    \verb+PATH=$PATH:<PATH_TO_ACSE_FOLDER>/bin+ Press \keys{Ctrl + X}, type Y and press \keys{Enter} to save.
  \end{enumerate}
\section{Usage}
\begin{enumerate}
  \item Open the Linux distro App and use \verb+cd+ to select the acse folder (\verb+cd acse_x.x.x/acse+, where x.x.x is the version number).
  \item Type \verb+code Acse.lex Acse.y+ to edit your language functionalities. This will open a new Visual Studio Code window under the Linux context. The output of the builtin Terminal emulator will be from Linux's \verb+bash+ and no longer from \verb+cmd+.
  \item When finished coding, move to the acse root folder (\verb+acse_x.x.x+) and enter the \verb+make+ command. \newline \newline \begin{small} Note: this is required \emph{every time} the source files (including \verb+Acse.lex+ and \verb+Acse.y+) are modified, since \verb+acse+ binary must be recompiled.\end{small}
\end{enumerate}
\section{Testing}
To test the compiler, it is suggested to follow the next steps:
\begin{enumerate}
  \item Create a folder into \verb+acse_x.x.x\tests+, create a \verb+.src+ file (containing the source code of the language to test) and place it into that folder.
  \item Add the name of the folder to the \verb+dirs+ variable into the \verb+tests\Makefile+.
  \item Run \verb+make+ to compile sources. This is requires every time a source file modification occurs.
\newline\newline \begin{small}Note: after \emph{every build}, remember to leave only the \verb+*.src+ file into the folder and delete evertything else. Otherwise \verb+make+ will skip that folder.\end{small}
\end{enumerate}
\end{document}