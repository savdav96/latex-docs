\documentclass{article}
\usepackage{graphicx}
\usepackage[os=win]{menukeys}

\begin{document}

\title{ACSE Quick installation guide}
\author{Davide Savoldelli}

\maketitle

\section{Requirements}
\begin{itemize}
    \item Windows 10 build 14393 or above.
    \item The ACSE devkit version 1.1.5 or above. This can be provided by your lecturer on the appropriate channels.
\end{itemize}

\section{Installation of Windows Subsystem for Linux (WSL)}
In order to provide a clean installation of the required binaries for the compiler to work, without the need to install 
a complete Virtual Machine on the System or to use the dual boot, the best option is to opt for
the WSL. Windows Subsystem for Linux is a lightweight version of a Linux distro under Windows 10 environment.
With it, the POSIX syscalls of the virtualized machine are on-the-fly translated into Windows NT system calls.
\begin{enumerate}
\item Press \keys{Win + R}, type 'powershell' and press \keys{\ctrl + \shift + \enter}. Insert the Administrator password if prompted.
    \item Type: 
    \begin{verbatim} 
    Enable-WindowsOptionalFeature -Online 
    -FeatureName Microsoft-Windows-Subsystem-Linux \end{verbatim} and press \keys{\enter}.
    \item Reboot the computer.
    \item Go to the Microsoft Store Application and install a Linux distro. \url{https://www.microsoft.com/it-it/p/pengwin/9nv1gv1pxz6p}{Pengwin} or \href{https://www.microsoft.com/it-it/p/ubuntu/9nblggh4msv6}{Ubuntu} are recommended.
  \end{enumerate}
\end{document}