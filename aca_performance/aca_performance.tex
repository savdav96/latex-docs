\documentclass[12pt]{article}
\usepackage{lingmacros}
\usepackage{tree-dvips}
\usepackage{titlesec}
\usepackage{amsmath}
\usepackage{hyperref}
\hypersetup{
    colorlinks,
    citecolor=black,
    filecolor=black,
    linkcolor=black,
    urlcolor=black
}



\setcounter{secnumdepth}{4}
\begin{document}
\title{%
  Advanced Computer Architectures Performance formulas \\
  \large Computer Science Engineer \\
  Politecnico di Milano \\
}

\author{Davide Savoldelli}

\maketitle
\tableofcontents
\newpage
\section{Speedup}        
    \subsection{Performance}
        \[Performance(x) = \frac{1}{Ex\:Time(x)}\]
    \subsection{Performance ratios}
        "X is n times faster than Y" means
        \[\frac{Ex\:Time(y)}{Ex\:Time(x)} = \frac{Performance(x)}{Performance(y)}\]
    \subsection{Performance percentage}
        "X is n \% times faster than Y" means
        \[\frac{Ex\:Time(y)}{Ex\:Time(x)} = 1 + \frac{n}{100}\]
        \[Speedup(x, y) = \frac{Performance(x)}{Performance(y)}\]
\section{Amdhal's law}
    \[Speedup(E) = \frac{Performance_{with}(E)}{Performance_{without}(E)} = \frac{Ex\:Time_{without}(E)}{Ex\:Time_{with}(E)}\]
    \[Ex\:Time_{with} = Ex\:Time_{without}\cdot[(1-Fraction_{enhanced}) + \frac{Fraction_{enhanced}}{Speedup_{enhanced}}]\]
    \[Speedup(E) = \frac{Ex\:Time_{without}(E)}{Ex\:Time_{with}(E)} = \frac{1}{(1-Fraction_{enhanced}) + \frac{Fraction_{enhanced}}{Speedup_{enhanced}}}\]
\newpage
\section{CPU Time - execution time}
    \subsection{CPU Time}
    \[CPU\:time = \frac{\#\:CC}{CC\:Frequency}\]
    \[CPU\:time = \#\:CC \cdot CC\:Time\]

    \[CPU\:time = \frac{Time}{Program} = \frac{\#Instructions}{Program} \cdot \frac{\#CC}{Instruction} \cdot \frac{Time}{CC}\]
    \[CPU\:time = IC \cdot CPI \cdot CC\:Time\]
    \newline
    \textbf{IC}: Instructions count, the number of instructions the program is composed, it depends on  program - obviously -, compiler and Instruction Set.
    \newline\newline
    \textbf{CPI}: Average Cycles per Instruction, this term is reduced with pipelining, it depends on Instruction Set and organization of the processor.
    \newline\newline
    \textbf{CC Time}: the time needed to perform a the Clock Cycle (CC), it is the inverse of Clock Frequency. It depends on organization of the processor and the technology used.


    \subsection{Average CPI}
    \[CPI = \frac{CPU\:time}{IC \cdot CC\:Time} = \frac{CPU\:Time\cdot CC\:Frequency}{IC}\]
    \[CPU\:time = CC\:Time \cdot \sum_{i=1}^{IC}{CPI_i \cdot \#I_i}\]
    \[CPI = \sum_{i=1}^{IC}{CPI_i \cdot F_i} \text{\footnotesize{    with F, instruction frequency   }} F_i = \frac{\#I_i}{IC}\]

\section{MIPS and MFLOPS}
    \subsection{MIPS}
    Millions of instructions per second
    \[MIPS = \frac{IC}{CPU\:time\cdot 10^6} = \frac{CC\:Frequency}{CPI \cdot 10^6}\]
    \[CPU\:time = \frac{IC}{MIPS\cdot 10^6} \]
    \subsection{MFLOPS}
    Millions of floating point operations per second
    \[MFLOPS = \frac{IC_{floating\:point}}{CPU\:time\cdot 10^6} \]

\section{Power consumption}
    \[P = \frac{E}{t}\]
    \[P = \frac{1}{CPU\:time}\cdot\sum_{i=1}^{n}{E_i \cdot I_i}\]

\section{Dependability}
    Mean time to failure (MTTF)\newline
    Mean time to repair (MTTR)\newline
    Mean time between failures (MTBF) \[MTBF = MTTF + MTTR\]
    Availability \[A = \frac{MTTF}{MTBF}\]

\end{document}